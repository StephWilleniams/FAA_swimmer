\documentclass{article}

\title{Passive particle spatial distribution can be controlled by modifying active particle boundary scattering}
\author{Stephen Williams}

\begin{document}

\maketitle

% Finish downloading mactex on chrome...

\section{Introduction}

% introduce the system and its physical relevance 
For microswimmers navigating their natural environments, it is common that they will encounter passive structures. % barriers and objects?
These interactions, depending on properties of the swimmer, lead to fundamentally different behaviours.
Some key examples of this include ... Choose 3 examples of this and references each: sperm/duct obviously, ideally something with microswimmer-free passive particle, and one with both (vibrio might actually be a nice example of this).
One key reasons for these differences lie in swimmer-boundary interactions.
It has been previously shown that the interactions of microswimmers with surfaces is dominated by ciliary contact events [kantsler 2013].
To attempt to capture the dynamics of these systems, a vast number of models have been proposed.
In [shape matters paper], an extensive outline of the existing models, experiments and theory is outlined.
These models generally comprise of a number of desirable features needed to capture the dynamics of microswimmers near to a boundary. 
In particular: random microswimmer motion, hydrodynamic interactions, and steric interactions both between each other and any boundaries.
In this paper, we will focus on swimmers which are composed of a fore-aft asymmetric ensemble of self propelled Particles (SPPs), following previous works by [lowens, kantsler].

\section{Active forces and torques}

WCA potential

\begin{equation} 
U = 4\epsilon \left( \frac{\sigma}{r}^12 - \frac{\sigma}{r}^6 \right)
\end{equation}

Derivative in r

\begin{equation} 
U = 4\epsilon \left( \frac{\sigma}{r}^12 - \frac{\sigma}{r}^6 \right)
\end{equation}

\end{document}
